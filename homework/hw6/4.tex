\chapter{Задание №3}
\section{Рисунки и таблицы}

% рис. 13
\begin{figure}[h]
    \includegraphics[scale=0.8]{fig_13}
    
    \caption{Окружности \\
    	а — на плоскости (поверхность нулевой гауссовой кривизны); б — иа по-
    	поверхности положительной гауссовой кривизны; в — на поверхности отрица-
    	отрицательной гауссовой кривизны}
    \label{eq:image_13}
\end{figure}

% полностью страница 22

Средняя кривизна поверхности входит в результаты многих и разнообразных механических задач.
Второй инвариант представляет собой так называемую \textit{гауссову} (или полную) \textit{кривизну} поверхности в данной точке:
\begin{displaymath}
    K=k_1k_2.
\end{displaymath}


В зависимости от знаков главных кривизн $k_1$ и $k_2$ можно отметить следующие характерные случаи в исследуемой точке, охватывающие любые ре
\begin{wrapfigure}[14]{l}{0.67\linewidth}
    \includegraphics[width=\linewidth]{fig_9}
    \caption{Отрицательная гауссова
        в точке поверхности
        кривизна}
    \label{eq:image_9}
\end{wrapfigure} гулярные поверхности. 


Если главные кривизны $k_1$ и $k_2$ имеют одинаковые знаки, то кривиз    
\begin{wrapfigure}{l}{2.2\linewidth}
    \includegraphics[width=\linewidth]{fig_10}
    \caption{Нулевая гауссова кривизна в точке
        поверхности}
    \label{eq:image_10}
\end{wrapfigure} на поверхности в исследуемой точке \textit{положительна} и сама поверхность в окрестности этой точки имеет вид, показанный на рис. 7 (индикатриса Дюпена в этом случае — эллипс).

Если главные кривизны $k_1$ и $k_2$ имеют разные знаки, то гауссова кривизна поверхности в исследуемой точке А \textit{отрицательна}, а сама поверхность в окрестности этой точки имеет седлообразный вид, изображенный на рис. \ref{eq:image_9} (индикатриса Дюпена представляет собой
в этом случае гиперболы).

Если одна из кривизн равна нулю, то гауссова кривизна в ис-
исследуемой точке А равна \textit{нулю} и поверхность
в окрестности этой точки имеет вид, изображенный на рис. \ref{eq:image_10} (индикатриса Дюпена представляет собой в этом случае две параллельные прямые).

Наконец, нулю гауссова кривизна может равняться в точке А
поверхности и в том случае, если обе главные кривизны равны
нулю. Такие точки называются \textit{точками уплощения}; в их окрестности поверхность имеет сложные свойства.

\newpage
\subsection{Из картинки}
\begin{table}[!hbp]
    \centering
    \caption{Перевозка бетонных и железобетонных изделий, стеновых и перегородочных материалов(кирпич, блоки, камни, плиты, панели), лесоматериалов круглых и пиломатериалов}
    \label{tab:table1}
    \begin{tabular}{| p{2,5cm} |c|c|c|c|}
        \hline
        \textbf{Расстояние перевозки, км} & \multicolumn{4}{|c|}{Класс груза} \\
        \cline{2-5}
        & \multicolumn{2}{|c|}{\textbf{1}} & \multicolumn{2}{|c|}{\textbf{2}} \\ 
        \cline{2-5}
        & \textbf{на 01.01.200 г.} & \textbf{на 01.03.2005 г.} & \textbf{ на 01.01.2000 г.} & \textbf{на 01.03.2005 г.} \\ \hline
        1 & \textbf{3,28} & 16,19 & \textbf{4,17} & 20,58 \\ 
        \hline
        2 & \textbf{4,17} & 20,58 & \textbf{5,21} & 25,71 \\ 
        \hline
        3 & \textbf{5,21} & 25,71 & \textbf{6,55} & 32,32 \\ 
        \hline
        4 & \textbf{6,26} & 30,89 & \textbf{7,74} & 38,20 \\ 
        \hline
        5 & \textbf{8,19} & 40,42 & \textbf{8,93} & 44,07 \\ 
        \hline
        6 & \textbf{9,22} & 45,50 & \textbf{10,27} & 50,68 \\ 
        \hline
        7 & \textbf{10,13} & 11,17 & \textbf{7,74} & 38,20\\ 
        \hline
        8 & \textbf{12,20} & 60,21 & \textbf{12,20} & 60,21\\ 
        \hline
        9 & \textbf{20,55} & 101,41 & \textbf{7,74} & 38,20\\ 
        \hline
        10 & \textbf{27,24} & 134,43 & \textbf{20,55} & 101,41\\ \hline
        20 & \textbf{ 33,19} & 163,79 &  \textbf{20,55} & 101,41 \\ \hline
        30 & \textbf{40,20} & 189,90 &  \textbf{ 33,19} & 163,79\\ \hline
        40 & \textbf{12,20} & 60,21 & \textbf{12,20} & 60,21\\ 
        \hline
        50 & \textbf{20,55} & 101,41 & \textbf{7,74} & 38,20\\ 
        \hline
        60 & \textbf{27,24} & 134,43 & \textbf{20,55} & 101,41\\ \hline
        70 & \textbf{ 33,19} & 163,79 &  \textbf{20,55} & 101,41 \\ \hline
        80 & \textbf{40,20} & 189,90 &  \textbf{ 33,19} & 163,79\\ \hline
        90 & \textbf{140,20} & 149,90 &  \textbf{ 33,19} & 163,79\\ \hline
        100 & \textbf{112,20} & 340,21 & \textbf{72,20} & 60,21\\ 
        \hline
        110 & \textbf{120,55} & 401,41 & \textbf{98,74} & 38,20\\ 
        \hline
        120 & \textbf{127,24} & 334,43 & \textbf{90,55} & 101,41\\ \hline
    \end{tabular}
\end{table}

\subsection{Из файла}
\begin{longtable}[!hbp]{l p{5cm} p{5cm} p{3cm}}        \caption{Перечень сборников государственных сметных норм на строительные и специальные строительные работы (ГЭСН-2001)}
    \label{tab:table2} \\
    \hline
    \textnumero Сборника	& Наименование сборника	& Полное обозначение сборника	& Сокращенное обозначение сборника \\ \hline
    1 	& Земляные работы							& ГЭСН 81-02-01-2001 	& ГЭСН-2001-01\\ 
    2 	& Горновскрышные работы						& ГЭСН 81-02-02-2001 	& ГЭСН-2001-02\\
    3 	& Буровзрывные работы						& ГЭСН 81-02-03-2001 	& ГЭСН-2001-03\\ 
    4 	& Скважины									& ГЭСН 81-02-04-2001 	& ГЭСН-2001-04\\
    5 	& Свайные работы. Закрепление грунтов. 
    Опускные колодцы						& ГЭСН 81-02-05-2001 	& ГЭСН-2001-05\\
    6 	& Бетонные и железобетонные 
    конструкции монолитные					& ГЭСН 81-02-06-2001 	& ГЭСН-2001-06\\ 
    7	& Бетонные и железобетонные 
    конструкции сборные						& ГЭСН 81-02-07-2001 	& ГЭСН-2001-07\\ 
    8	& Конструкции из кирпича и блоков 			& ГЭСН 81-02-08-2001 	& ГЭСН-2001-08\\ 
    9	& Строительные металлические 
    конструкции 							& ГЭСН 81-02-09-2001 	& ГЭСН-2001-09\\ 
    10 	& Деревянные конструкции					& ГЭСН 81-02-10-2001 	& ГЭСН-2001-10\\ 
    11 	& Полы										& ГЭСН 81-02-11-2001 	& ГЭСН-2001-11\\ 
    12 	& Кровли									& ГЭСН 81-02-12-2001 	& ГЭСН-2001-12\\ 
    13 	& Защита строительных конструкций 
    и оборудования от коррозии				& ГЭСН 81-02-13-2001 	& ГЭСН-2001-13\\ 
    14 	& Конструкции в сельском строительстве		& ГЭСН 81-02-14-2001 	& ГЭСН-2001-14\\ 
    15 	& Отделочные работы							& ГЭСН 81-02-15-2001 	& ГЭСН-2001-15\\ 
    16 	& Трубопроводы внутренние					& ГЭСН 81-02-16-2001 	& ГЭСН-2001-16\\ 
    17 	& Водопровод и канализация --- 
    внутренние устройства					& ГЭСН 81-02-17-2001 	& ГЭСН-2001-17\\ 
    18 	& Отопление --- внутренние устройства		& ГЭСН 81-02-18-2001 	& ГЭСН-2001-18\\ 
    19 	& Газоснабжение --- внутренние устройства	& ГЭСН 81-02-19-2001 	& ГЭСН-2001-19\\ 
    20 	& Вентиляция и кондиционирование воздуха	& ГЭСН 81-02-20-2001 	& ГЭСН-2001-20\\
    21 	& Временные сборно-разборные здания 
    и сооружения							& ГЭСН 81-02-21-2001 	& ГЭСН-2001-21\\ 
    22 	& Водопровод --- наружные сети				& ГЭСН 81-02-22-2001 	& ГЭСН-2001-22\\ 
    23 	& Канализация --- наружные сети				& ГЭСН 81-02-23-2001 	& ГЭСН-2001-23\\ 
    24 	& Теплоснабжение и газопроводы				& ГЭСН 81-02-24-2001 	& ГЭСН-2001-24\\ 
    25 	& Магистральные и промысловые трубопроводы	& ГЭСН 81-02-25-2001 	& ГЭСН-2001-25\\ 
    26 	& Теплоизоляционные работы					& ГЭСН 81-02-26-2001 	& ГЭСН-2001-26\\ 
    27 	& Автомобильные дороги						& ГЭСН 81-02-27-2001 	& ГЭСН-2001-27\\ 
    28 	& Железные дороги							& ГЭСН 81-02-28-2001 	& ГЭСН-2001-28\\ 
    29 	& Тоннели и метрополитены					& ГЭСН 81-02-29-2001 	& ГЭСН-2001-29\\ 
    30 	& Мосты и трубы								& ГЭСН 81-02-30-2001 	& ГЭСН-2001-30\\ 
    31 	& Аэродромы									& ГЭСН 81-02-31-2001 	& ГЭСН-2001-31\\ 
    32 	& Трамвайные пути							& ГЭСН 81-02-32-2001 	& ГЭСН-2001-32\\ 
    33 	& Линии электропередачи						& ГЭСН 81-02-33-2001 	& ГЭСН-2001-33\\ 
    34 	& Сооружения связи, радиовещания 
    и телевидения							& ГЭСН 81-02-34-2001 	& ГЭСН-2001-34\\ 
    35 	& Горнопроходческие работы					& ГЭСН 81-02-35-2001 	& ГЭСН-2001-35\\ 
    36 	& Земляные конструкции гидротехнических 
    сооружений								& ГЭСН 81-02-36-2001 	& ГЭСН-2001-36\\ 
    37 	& Бетонные и железобетонные конструкции 
    гидротехнических сооружений				& ГЭСН 81-02-37-2001 	& ГЭСН-2001-37\\ 
    38 	& Каменные конструкции гидротехнических 
    сооружений								& ГЭСН 81-02-38-2001 	& ГЭСН-2001-38\\ 
    39 	& Металлические конструкции 
    гидротехнических сооружений				& ГЭСН 81-02-39-2001 	& ГЭСН-2001-39\\ 
    40 	& Деревянные конструкции гидротехнических 
    сооружений								& ГЭСН 81-02-40-2001 	& ГЭСН-2001-40\\ 
    41 	& Гидроизоляционные работы в 
    гидротехнических сооружениях			& ГЭСН 81-02-41-2001 	& ГЭСН-2001-41\\ 
    42 	& Берегоукрепительные работы				& ГЭСН 81-02-42-2001 	& ГЭСН-2001-42\\ 
    43 	& Судовозные пути стапелей и слипов			& ГЭСН 81-02-43-2001 	& ГЭСН-2001-43\\ 
    44 	& Подводностроительные (водолазные) работы	& ГЭСН 81-02-44-2001 	& ГЭСН-2001-44\\ 
    45 	& Промышленные печи и трубы					& ГЭСН 81-02-45-2001 	& ГЭСН-2001-45\\ 
    46 	& Работы по реконструкции зданий 
    и сооружений							& ГЭСН 81-02-46-2001 	& ГЭСН-2001-46\\ 
    47 	& Озеленение. Защитные лесонасаждения		& ГЭСН 81-02-47-2001 	& ГЭСН-2001-47\\ 
    48 	& Скважины на нефть и газ					& ГЭСН 81-02-48-2001 	& ГЭСН-2001-48\\ 
    49 	& Скважины на нефть и газ 
    в морских условиях						& ГЭСН 81-02-49-2001 	& ГЭСН-2001-49\\ 
\end{longtable}	
